\documentclass[a4paper,12pt]{article}
\usepackage{amsmath,amssymb}
\begin{document}
	\title{Homework 4}
	\author{Sean Papay\\
		007323511
		\and
		Dakota Polenz\\
		007879664
		\and
		Ryan Lichtig\\
		007264348
	}
	\date{}
\maketitle
	\begin{description}
		\item[Problem 9.10] \hfill \\
		The riddle can be expressed in first-order logic as:\\
		$Male(m) \wedge$\\
		$(\neg \exists s:Brother(s, I) \vee Sister(s, I)) \wedge
		(\exists f:Son(f, Father(I)) \wedge f = Father(m))$ \\
		Where $I$ represents the speaker in the riddle, and $m$ represents ``that man''.
		From this, we can use the axioms of the kinship domain to deduce the identity
		of ``that man'':
		\begin{equation}
		\begin{split}
			Male(m) \wedge \\
			(\neg \exists s:Brother(s, I) \vee Sister(s, I)) \wedge \\
			(\exists f:Son(f, Father(I)) \wedge f = Father(m)) 
		\end{split}
		\end{equation}
		
		\begin{equation}
		\begin{split}
			Male(m) \wedge \\
			(\neg \exists s:(Sibling(s, I) \wedge Male(s)) \vee (Sibling(s, I) \wedge Female(s)) \wedge \\
			(\exists f:Son(f, Father(I)) \wedge f = Father(m)) 
		\end{split}
		\end{equation}
		
		
		\begin{equation}
		\begin{split}
			Male(m) \wedge \\
			(\neg \exists s:Sibling(s, I) \wedge (Male(s) \vee Female(s))) \wedge \\
			(\exists f:Son(f, Father(I)) \wedge f = Father(m)) 
		\end{split}
		\end{equation}
		
		\begin{equation}
		\begin{split}
			Male(m) \wedge \\
			(\neg \exists s:Sibling(s, I)) \wedge \\
			(\exists f:Son(f, Father(I)) \wedge f = Father(m)) 
		\end{split}
		\end{equation}
		
		\begin{equation}
		\begin{split}
			Male(m) \wedge \\
			(\neg \exists s:Sibling(s, I)) \wedge \\
			(\exists f:Son(f, Father(I)) \wedge Parent(f, m) \wedge Male(f)) 
		\end{split}
		\end{equation}
		
		\begin{equation}
		\begin{split}
			Male(m) \wedge \\
			(\neg \exists s:Sibling(s, I)) \wedge \\
			(\exists f:Child(f, Father(I)) \wedge Parent(f, m) \wedge Male(f)) 
		\end{split}
		\end{equation}
		
		\begin{equation}
		\begin{split}
			Male(m) \wedge \\
			(\neg \exists s: s \neq I \wedge(\exists p: Parent(p, s) \wedge Parent(p, I))) \wedge \\
			(\exists f:Child(f, Father(I)) \wedge Parent(f, m) \wedge Male(f)) 
		\end{split}
		\end{equation}
		
		\begin{equation}
		\begin{split}
			Male(m) \wedge \\
			(\forall s: s = I \vee \neg (\exists p: Parent(p, s) \wedge Parent(p, I))) \wedge \\
			(\exists f:Child(f, Father(I)) \wedge Parent(f, m) \wedge Male(f)) 
		\end{split}
		\end{equation}
		
		\begin{equation}
		\begin{split}
			Male(m) \wedge \\
			(\forall s: s = I \vee \neg (\exists p: Parent(p, s) \wedge Parent(p, I))) \wedge \\
			(\exists f, g:g = Father(I) \wedge Child(f, g) \wedge Parent(f, m) \wedge Male(f)) 
		\end{split}
		\end{equation}
		
		\begin{equation}
		\begin{split}
			Male(m) \wedge \\
			(\forall s: s = I \vee \neg (\exists p: Parent(p, s) \wedge Parent(p, I))) \wedge \\
			(\exists f, g:Parent(g, I) \wedge Child(f, g) \wedge Parent(f, m) \wedge Male(f) \wedge Male(g)) 
		\end{split}
		\end{equation}
		
		\begin{equation}
		\begin{split}
			Male(m) \wedge \\
			(\forall s: s = I \vee \neg (\exists p: Parent(p, s) \wedge Parent(p, I))) \wedge \\
			(\exists f, g:Parent(g, I) \wedge Parent(g, f) \wedge Parent(f, m) \wedge Male(f) \wedge Male(g)) 
		\end{split}
		\end{equation}
		
		\begin{equation}
		\begin{split}
			Male(m) \wedge \\
			(\forall s: (\exists p: Parent(p, s) \wedge Parent(p, I)) \rightarrow s = I) \wedge \\
			(\exists f, g:Parent(g, I) \wedge Parent(g, f) \wedge Parent(f, m) \wedge Male(f) \wedge Male(g)) 
		\end{split}
		\end{equation}
		
		\begin{equation}
		\begin{split}
			Male(m) \wedge \\
			(\forall s: (\exists p: Parent(p, s) \wedge Parent(p, I)) \rightarrow s = I) \wedge \\
			(\exists f: f = I \wedge Parent(f, m) \wedge Male(f)) 
		\end{split}
		\end{equation}
		
		\begin{equation}
		\begin{split}
			Male(m) \wedge Parent(I, m) \wedge Male(I) 
		\end{split}
		\end{equation}
		
		\begin{equation}
		\begin{split}
			Male(m) \wedge Child(m, I)
		\end{split}
		\end{equation}
		
		\begin{equation}
		\begin{split}
			Son(m, I)
		\end{split}
		\end{equation}
		
		From this, we can conclude that ``that man'' is, in fact, the speaker's son.\\
		\\
		In prenex normal form, the original riddle can be expressed as follows:\\
		
		$\exists f : \forall s: (\neg Brother(s, I)\vee
		\neg Son(f, Father(I)) \vee \neg f = Father(m) \vee \neg Male(m)) \wedge (\neg Sister(s, I) \vee
		\neg Son(f, Father(I)) \vee \neg f = Father(m) \vee \neg Male(m)) $
		
	\item[Problem 9.23] \hfill
	\begin{description}
		\item[a.]
		Premise: $\forall h: Horse(h) \rightarrow Animal(h)$\\
		Conclusion: $\forall c, q: (HeadOf(c, q) \wedge Horse(q)) \rightarrow 
		\exists a: (HeadOf(c, a) \wedge Animal(a))$
		
		\item[b.]
		Premise: $\neg Horse(h) \vee Animal(h)$\\
		Conclusion: $HeadOf(C, Q) \wedge Horse(Q) \wedge 
		(\neg HeadOf(C, a) \vee \neg Animal(a))$
		\item[c.]
		$(\neg HeadOf(C, Q) \vee \neg Animal(Q)), HeadOf(C, Q) \over \neg Animal(Q)$ \\
		\\
		$\neg Animal(Q), (\neg Horse(h) \vee Animal(h)) \over \neg Horse(Q) $\\
		\\
		$\neg Horse(Q), Horse(Q) \over \bot$
		\end{description}
	
	\item[Additional Problem 1]
	\end{description}
\end{document}
